\chapter{Fazit}
Rückblickend ist zu sagen, dass der Versuch eine gute Möglichkeit ist, den Kondensator näher zu untersuchen. Es ist durch
den Aufbau vergleichsweise einfach, Kapazität, elektrische Feldkonstante und Dielektrika zu ermitteln.\par
Anhand des Diagramms in \bild{fig:farad} ist zu erkennen, dass die drei Messreihen ähnliche Ergebnisse für die Kapazität
geliefert haben. Der tatsächlich berechnete Wert liegt ebenfalls in unmittelbarer Nähe der anderen Punkte. Alle Punkte
werden durch die jeweiligen Fehlerbalken gedeckt. Der größte Teil, der den Fehler ausmacht, ist vermutlich die Spannung
$ U_{2} $, da der Zeiger des Messgerät immer weiter ausgeschlagen hat. Als Abhilfe wurde die Spannung unmittelbar nach
Betätigen des Umschalters abgelesen.\par
Der Fehler des berechneten Kapazitätswerts hängt von der Abweichung des Durchmessers und des Abstandes der Kondensatorplatten
ab. Da der Durchmesser schwer zu messen war, wurde mit $ \Delta D=\SI{0,5}{cm} $ eine relativ hohe Ungenauigkeit gewählt.
Noch höher gewählt wurde die Abstandsungenauigkeit mit $ \Delta d=\SI{1}{mm} $, da die Skala nur auf ein Millimeter genau
abgelesen wurde, jedoch durch die Feinverstellung die Ungenauigkeit viel geringer ist als ein Millimeter.\par
Die Bestimmung der elektrischen Feldkonstante war ebenfalls unkompliziert und führte mit
$ \varepsilon_{0}=(8,4\pm 1,7){\cdot 10^{-12}\frac{\text{As}}{\text{Vm}}} $ zu einem plausiblen Ergebnis. Die Abweichung
deckt sich mit dem Literaturwert für \(\varepsilon_0\) von \( \SI{8,854}{\cdot 10^{-12} \frac{As}{Vm}}\) \cite{Haberle.2007}.
Durch das
Diagramm in \bild{fig:mu_0} ist die Linearität gut zu erkennen. Jedoch erscheinen die Fehlerbalken aufgrund der Fehlerfortpflanzung
mit abnehmendem Abstand sehr groß. Der Fehler der elektrischen Feldkonstante setzt sich zusammen aus der Durchmesserabweichung,
welche nur einen kleinen Teil des Gesamtfehlers ausmacht, und aus dem mit Hilfe von \textit{SciDAVis} ermittelten Fehler
der Steigung \(m\).
Das hängt vermutlich mit dem wie bereits erwähnten und zu hoch gewählten Abstandsfehler zusammen.\par
Die Aufnahme der Messwerte für die Bestimmung der Dielektrizitätszahl war zwar einfach durchzuführen, brachte dennoch einige
kleine Probleme mit sich. Beim Einführen und Herausnehmen der Platten wurde vermutlich der Abstand ein paar Mal
ungewollt verändert, was sich vor allem bei größeren Spannungen $ U_{2} $ beim Pertinax erkenntlich gemacht hat. Das
Ablesen der Spannung $ U_{2} $ wurde durch das Umstellen und Umdenken der Größenordnung am Messgerät erschwert. Als
Ergebnis kamen mit $ \varepsilon_{r,Plexiglas}=4,05\pm 0,15 $ und $ \varepsilon_{r,Pertinax}=8,73\pm 0,92 $ dennoch
Werte mit (relativ) kleinen Abweichungen heraus. Die Mittelwerte und Standartabweichungen der Messreihen in \tabelle{tab:mess2}
verdeutlichen, wie genau die Messungen tatsächlich waren.\par
Trotz kleinerer Probleme ist der Versuch nichts desto trotz gut geeignet für die Behandlung der Komponenten eines
Kondensators und führte zu erwarteten Resultaten.
\chapter{Versuchsdurchführung}
Zunächst soll der Versuchsaufbau wie in \bild{fig:schematic_aufbau} geschaltet werden. Dabei ist das Messgerät Unigor 3 für die
Messung von $ U_{1} $ zu verwenden. Der Messverstärker soll je nach Spannung auf den Faktor $ 10^{0} $ oder $ 10^{1} $
eingestellt werden.
%
\section{Bestimmung der Kapazität des variablen Kondensators}
In diesem Versuchsabschnitt soll die Kapazität $ C_{1} $ bestimmt werden, indem die Spannung $ U_{2} $ für
\( U_{1}=\SI{250}{V}\), \(\SI{500}{V}\) und \(\SI{1000}{V}\) jeweils mit den Plattenabständen
$ \SI{1}{mm} \leq d \leq \SI{10}{mm} $ in $ \SI{1}{mm} $ Schritten gemessen wird. Der Umschalter befindet sich hierzu
in Stellung 2-1, sodass der Kondensator $ C_{1} $ aufgeladen wird.\par
Zuerst ist das Hochspannungsnetzteil einzuschalten und auf die jeweilige Spannung einzustellen. Anschließend soll der
Plattenabstand justiert werden. Danach soll der Umschalter S1 in Stellung 2-3 geschaltet werden, die Spannung $ U_{2} $
am Messgerät \textit{Unigor 1} abgelesen und notiert werden. Daraufhin soll der Schalter S1 wieder in die Ursprungsstellung
zurückgestellt und anschließend Taster betätigt werden.\par
Diese Schritte sollen für jeden weiteren Abstand \(d\) und jede neue Spannung \(U_1\) durchgeführt werden.\par
%
\section{Bestimmung der Dielektrizitätszahl}
Nun sollen die Dielektrizitätszahlen der beiden zur Verfügung gestellten Platten ermittelt werden. Dazu soll die Spannung
$ U_2 $ 10 mal je Platte bei $ U_1 = \SI{1000}{V} $ abwechselnd mit und ohne Platte gemessen werden. Der Schaltkreis
bleibt der in \bild{fig:schematic_aufbau} gezeigte.\par
Zu Beginn soll das Hochspannungsnetzteil auf $ \SI{1000}{V} $ eingestellt und die Platte zwischen den beiden Elektroden platziert
werden, so, dass sie ohne Anstrengung wieder herausgenommen werden kann. Der Umschalter S1 soll dann so geschaltet sein,
sodass der Kondensator $ C_{1} $ aufgeladen wird, woraufhin der Taster zum entleeren des Vergleichskondensators gedrückt
werden soll. Nachdem der Umschalter nochmals betätigt werden soll, ist nun die Spannung $ U_{2} $ abzulesen und zu notieren.
Das ganze soll abwechselnd mit und ohne Platte durchgeführt werden.\par